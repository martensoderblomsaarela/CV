%-*- program: xelatex -*-        
%-*- program: biber -*-`        
%-*- program: xelatex -*-
%------------------------------------
% Dario Taraborelli
% Typesetting your academic CV in LaTeX
%
% URL: http://nitens.org/taraborelli/cvtex
% DISCLAIMER: This template is provided for free and without any guarantee 
% that it will correctly compile on your system if you have a non-standard  
% configuration.
% Some rights reserved: http://creativecommons.org/licenses/by-sa/3.0/
%------------------------------------

%!TEX TS-program = xelatex
%!TEX encoding = UTF-8 Unicode

\documentclass[10pt,a4paper]{article}
\usepackage{fontspec} 

\usepackage{setspace}
%\usepackage{xltxtra} %This package is currently causing problems, but it is important as it enables spacing of CJK punctuation marks.
\usepackage[fallback]{xeCJK}

\newcommand{\mainfontCJK}[0]{HanaMinA}


\setCJKmainfont[Scale=0.9%86
]
{\mainfontCJK}
\setCJKfallbackfamilyfont{\CJKrmdefault}[%Scale=0.9%85
]
{NanumMyeongjo}
\XeTeXlinebreaklocale "zh"
\XeTeXlinebreakskip = 0pt plus 1pt




% DOCUMENT LAYOUT
\usepackage{geometry} 
\geometry{a4paper, 
%letterpaper,
 textwidth=5.5in, textheight=9in%8.5in
, marginparsep=7pt, marginparwidth=.6in}
\setlength\parindent{0in}

% FONTS
\usepackage[usenames,dvipsnames]{color}
\usepackage{xunicode}
\usepackage{xltxtra}
\defaultfontfeatures{Mapping=tex-text}
\setromanfont [Ligatures={Common}, Numbers={OldStyle}, Variant=01]{Linux Libertine} %Linux Libertine O
\setmonofont[Scale=0.8]{Linux Libertine Mono} %Linux Libertine Mono O

% ---- CUSTOM COMMANDS
\chardef\&="E050
\newcommand{\html}[1]{\href{#1}{\scriptsize\textsc{[html]}}}
\newcommand{\pdf}[1]{\href{#1}{\scriptsize\textsc{[pdf]}}}
\newcommand{\doi}[1]{\href{#1}{\scriptsize\textsc{[doi]}}}
% ---- MARGIN YEARS
\usepackage{marginnote}
\newcommand{\amper{}}{\chardef\amper="E0BD }
\newcommand{\years}[1]{\marginnote{\scriptsize #1}}
\renewcommand*{\raggedleftmarginnote}{}
\setlength{\marginparsep}{7pt}
\reversemarginpar

% HEADINGS
\usepackage{sectsty} 
\usepackage[normalem]{ulem} 
\sectionfont{\mdseries\upshape\Large}
\subsectionfont{\mdseries\scshape\normalsize} 
\subsubsectionfont{\mdseries\upshape\large} 

% PDF SETUP
% ---- FILL IN HERE THE DOC TITLE AND AUTHOR
\usepackage[%dvipdfm, %<---- This was not working so I disabled it (6/4/2013)
bookmarks, colorlinks, breaklinks, 
% ---- FILL IN HERE THE TITLE AND AUTHOR
	pdftitle={Mårten Söderblom Saarela - CV},
	pdfauthor={Mårten Söderblom Saarela},
	pdfproducer={}
]{hyperref}  
\hypersetup{linkcolor=MidnightBlue,citecolor=MidnightBlue,filecolor=black,urlcolor=MidnightBlue%
} 

\usepackage{comment}
\excludecomment{exclude}

% DOCUMENT
\begin{document}
{\LARGE Mårten \textbf{Söderblom Saarela} 馬騰}\\[0.5cm]
%\begin{exclude}
Room 1402\\
Institute of Modern History, Academia Sinica\\ 
128 Academia Rd, Sec. 2\\
Nankang, Taipei 115, Taiwan\\[.2cm]
台灣台北市11529南港區研究院路二段128號中央研究院近代史研究所1402研究室\\[.2cm]
Phone: 	+886-2-27898-236\\[.2cm]
\noindent Email: \href{mailto:saarela@gate.sinica.edu.tw}{saarela@gate.sinica.edu.tw} %\begin{tabular}{l}\href{mailto:marten.soderblom.saarela@gmail.com}{marten.soderblom.saarela@gmail.com}\tabularnewline \href{mailto:msaarela@mpiwg-berlin.mpg.de}{msaarela@mpiwg-berlin.mpg.de}\end{tabular}\\

%\textsc{url}: \href{http://www.mpiwg-berlin.mpg.de/en/users/msaarela}{http://www.mpiwg-berlin.mpg.de/en/users/msaarela}\\ 
%\vfill


%Add this for European CV:
%Born:  September 5, 1985---Stockholm, Sweden\\
%Nationality:  Swedish


%\hrule
%\section*{Areas of specialization} Chinese history • Qīng empire (1636--1911) • Intellectual and cultural history • History of language studies • History of the book% • Chosŏn Korea (1392--1897)



%%%\hrule
%\section*{Appointments held}
%\noindent

%\hrule
%\begin{exclude}
\section*{Current position% 现任职务
}
\noindent \years{2019--}Assistant Research Fellow% (tenure track)
, Institute of Modern History, Academia Sinica\\ 中央研究院近代史研究所助研究員


%\hrule
\section*{Education% 教育状况
}

\noindent
\years{2010--15}\textsc{p}h\textsc{d} %博士 
East Asian Studies, Princeton University% 普林斯顿大学
\\[.2cm]
\emph{Dissertation:}\\
``Manchu and the Study of Language in China (1607--1911)''\\
%(awarded the Buchanan Dissertation Prize, see \hyperref[GrantsHonorsAndAwards]{below})\\
\emph{Committee members:}\\
Benjamin A.\ Elman, Susan Naquin, Anthony T.\ Grafton (Princeton); Nicola Di Cosmo (Institute for Advanced Study); Mark C.\ Elliott (Harvard)\\[.2cm]
\emph{Examination fields:}\\
``Late Imperial China'' (major, Benjamin A.\ Elman), ``Pre-Modern Korea'' (minor, Joy S.\ Kim), ``Early Modern Japan'' (minor, Federico Marcon)\\[.2cm]
\years{2013}\textsc{ma} %硕士
 East Asian Studies, Princeton University% 普林斯顿大学
 \\
\years{2008--09}\textsc{ma} Sinology, \textsc{soas} (London)% 伦敦大学亚非学院
\\
\years{2005--07}\textsc{ba} French (Chinese minor), Stockholm University% 斯德哥尔摩大学
%\years{2001--04}Kärrtorps gymnasium (high school), Stockholm

\subsection*{Other training% 其他培训
}
\noindent
\years{2014}Academia Sinica %台北中央研究院 
Winter Institute (``Into the Core of Traditional Chinese Scholarship''), January\\
\years{2013--14}Senior Visiting Student, Peking University% 北京大学
\\
\years{2012--13}Exchange Scholar Program, Harvard University% 哈佛大学
\\
\years{2011}Second year Japanese language course, Hokkaido International Foundation (Hakodate), June--August\\
\years{2010}First year Japanese language course, Beloit College (Wisconsin)% 伯洛伊特学院
, June--August\\
\years{2010}Exchange student in Korean, Hankuk University of Foreign Studies (Seoul)% 首尔韩国外国语大学
, January--June\\
\years{2009}Non-degree studies in Korean, Stockholm University% 斯德哥尔摩大学
, August--December\\
\years{2007--08}Non-degree advanced course in Mandarin Chinese, Tongji University (Shanghai)% 同济大学
\\
\years{2004--05}Undergraduate studies in French, Université de la Sorbonne-Nouvelle--Paris 3% 巴黎第三大学
\\
\years{2004}French language, Université Charles-de-Gaulle--Lille 3% 法国里尔第三大学
, January--June%\\

\section*{Appointments held or offered
}
\noindent \years{2015--18}Postdoctoral Fellow, Max Planck Institute for the History of Science% 马克斯·普朗克·科学史研究所
 , Berlin\\
\noindent \years{2018}W1 Junior Professorship for State and Society in China, \textsc{fau} Erlangen-Nürnberg\\ (6-year position; declined)\\
\noindent \years{2018}Associate Professor 副教授, Institute of China Studies, Zhejiang University International Campus (declined)\\
\years{2015}Postdoctoral Scholar-Fellow, Center for Chinese Studies, \textsc{uc} %University of California, 
Berkeley (declined)

\section*{Research affiliations% 访问
}
\years{2017}Visiting Scholar, Chinese Academy of Social Sciences% 中国社会科学院
, December\\
\years{2014}Special Researcher, Kyujanggak, Seoul National University% 国立首尔大学
, May--June\\
\years{2014}Visiting Fellow, Institute for Advanced Study on Asia, University of Tokyo% 东京大学
, April--May%\\

%\hrule
%\section*{Research affiliations}
%\noindent


%\end{exclude}




%\hrule

%\begin{exclude}


%These texts were used to get me hired at 近史所, cannot be used as 代表作 for promotion to Associate:
%甲、	Alphabets avant la lettre: Phonographic Experiments in Late Imperial China
%乙、	The Early Modern Travels of Manchu*(正在檢查中,還沒與出版社的合同)
%丙、	Mandarin over Manchu: Court-Sponsored Qing Lexicography and Its Subversion in Korea and Japan
%丁、	“Shooting Characters”: A Phonological Game and Its Uses in Late Imperial China (已經通過了檢查,尚未出版)
%戊、	The Manchu script and information management


\section*{%List of publications%
Publications}

\subsection*{Books}

%\noindent \years{\hspace{0.1in}。}
\noindent \years{2020}\emph{The Early Modern Travels of Manchu: A Script and Its Study in East Asia and Europe} (Philadelphia: University of Pennsylvania Press, in press)

\subsection*{Journal articles% 学刊的文章
}

\noindent \years{2019}``A Guide to Mandarin, in Manchu: On a Partial Translation of \emph{Guanhua zhinan} (1882) and Its Historical Context,'' \emph{East Asian Publishing and Society} 9: 1--28 

\noindent \years{2018}```Shooting Characters': A Phonological Game and Its Uses in Late Imperial China,'' \emph{Journal of the American Oriental Society} 138.2: 327--59 

\noindent \years{2017}``Mandarin over Manchu: Court-Sponsored Qing Lexicography and Its Subversion in Korea and Japan,'' \emph{Harvard Journal of Asiatic Studies} 77.2: 363--406

\noindent \years{2016}``Alphabets \emph{avant la lettre}: Phonographic Experiments in Late Imperial China,'' \emph{Twentieth-Century China} 41.3: 234--57 

\noindent \years{2014}``\emph{Shier zitou jizhu} (Collected notes on the twelve heads): A Recently Discovered Work by Shen Qiliang,'' \emph{Saksaha} 12: 9--31

\noindent \years{2014}``Shape and Sound: Organizing Dictionaries in Late Imperial China,'' \emph{Dictionaries: Journal of the Dictionary Society of North America} 35: 187--208

\noindent \years{2014}``The Qing Tradition and the Return of Manchu Lexicography to China (1970s--1990s): The Example of Alphabetical Order,'' \emph{Historiographia Linguistica} 41.2/3: 323--54

\noindent \years{2014}```Shi'er zitou' yu Qingdai Manwen yuxue'' 《十二字头》与清代满文语学 (The ``twelve heads'' and Manchu language studies in the Qing period), \emph{Qingshi yanjiu} 清史研究 3: 1--11

\noindent \years{2010}``Scholarly Discourse in Chen Li's (1810--1882) Letters,'' \emph{Sungkyun Journal of East Asian Studies} 10.2: 169--89

\subsection*{Book chapters% 论文集的文章
}

%\noindent \years{2019?}``A Guangxu Renaissance? Manchu Language Studies in the Late Qing and Their Republican Afterlife,'' forthcoming in \emph{Time, Language, and Power in Late Imperial and Republican China}, edited by Ori Sela, Zvi Ben-Dor Benite, and Joshua Fogel (Leiden: Brill) 

\noindent \years{2019?}``Lexicography of the Entrenched Empire: Banihûn's and Pu-gong's \emph{Manchu-Chinese Literary Ocean} (1821),'' forthcoming in \emph{The Whole World in a Book: Dictionaries in the Nineteenth Century}, edited by Sarah Ogilvie and Gabriella Safran (Oxford: Oxford University Press) 

\noindent \years{2019}``The Chinese Periphery to c.\ 1800,'' in \emph{Cambridge World History of Lexicography}, edited by John Considine (Cambridge: Cambridge University Press, in press) 

\noindent \years{2014}``The Manchu Script and Information Management: Some Aspects of Qing China's Great Encounter with Alphabetic Literacy,'' in \emph{Rethinking East Asian Languages, Vernaculars, and Literacies, 1000--1919}, edited by Benjamin A.\ Elman, 169--97 (Leiden: Brill)

\subsection*{Book reviews% 书评
}
\noindent \years{2019}Review of \emph{China's Philological Turn: Scholars, Textualism, and the Dao in the Eighteenth Century} by Ori Sela, \emph{History of Humanities} 4.2, in press

\noindent \years{2012}Review of \emph{China in European Encyclopaedias, 1700--1850} by Georg Lehner, \emph{Sungkyun Journal of East Asian Studies} 12.1: 96--9



%\subsection*{Texts for a general audience}

%\noindent \years{2019}``The Foreign Entanglements of Mandarin Chinese in the Eighteenth and Nineteenth Centuries,'' \href{https://hiphilangsci.net/2019/05/29/foreign-entanglements-mandarin-chinese/}{History and Philosophy of the Language Sciences}, May 29

%\noindent \years{2018}``A Look in the 1708 \emph{Mirror},'' \emph{Debtelin} 2:67--71

%\noindent \years{2017}``One Manchu Bibliographer Dates the Work of Another, Or `The Librairie Française and the Manchu books at Capital Library, Beijing,' cont.'' \emph{Manchu Studies Group} \href{http://www.manchustudiesgroup.org/2018/03/14/one-manchu-bibliographer-dates-the-work-of-another-or-the-librairie-francaise-and-the-manchu-books-at-capital-library-beijing/}{blog} (\href{http://www.manchustudiesgroup.org}{http://www.manchustudiesgroup.org})

%\noindent \years{2014}``The Librairie Française and the Manchu books at Capital Library, Beijing''\\ \emph{Manchu Studies Group} \href{http://www.manchustudiesgroup.org/2014/03/06/the-librairie-francaise-and-the-manchu-books-at-capital-library-beijing/}{blog}

%\noindent \years{2013}``Thoughts on the Rise and Fall of the Manchu Language,'' \emph{Manchu Studies Group}  \href{http://www.manchustudiesgroup.org/2013/04/29/thoughts-on-the-rise-and-fall-of-the-manchu-language/}{blog}

%\noindent \years{2013}``The Cost of a Manchu Dictionary in the Guangxu Period,'' \emph{Manchu Studies Group} \href{http://www.manchustudiesgroup.org/2013/03/04/715/}{blog}

%\subsection*{Translations}

%\noindent \years{2018}Contributing translator (Manchu to English) of entries from \emph{Imperially commissioned Mirror of the Manchu language} (1708), \emph{Debtelin} 2

\begin{exclude}

\section*{Works in progress}

%\hspace{0.1in}。

%\subsection*{Book manuscript}


\subsection*{Co-edited volumes% 合编的论文集
}

\noindent \years{\hspace{0.1in}。}With Glenn Most and Dagmar Schäfer, \emph{Coping with Plurilingualism} (working title), reader on multilingualism in historical societies for use in undergraduate teaching (chapters in progress, book not yet under contract)

\noindent \years{\hspace{0.1in}。}With Henning Klöter, \emph{Language Diversity in the Sinophone World} (working title), on multilingualism and historical sociolinguistics (papers in the final stages of editing, book not yet under contract)

%\noindent \years{\hspace{0.1in}。}With Dagmar Schäfer and Glenn Most, edited volume with source texts on multilingualism in history

\subsection*{Articles}

\noindent \years{\hspace{0.1in}。}``The Use of Manchu in Qing Official Communications in the Eighteenth Century, Or, Bilingualism in Three Phases,'' accepted for publication in \emph{Late Imperial China} conditioned upon substantial revision (July 20, 2019)

\noindent \years{\hspace{0.1in}。}``Joshua Marshman Studies Chinese and Proclaims a Sanskrit-Infused Sinology,'' under review as of April 17, 2019 %can be used for the promotion file

\noindent \years{\hspace{0.1in}。}``Public Inscriptions and Manchu Language Reform in the Early Qianlong Reign (1740s--60s),'' under review as of June 11, 2019

\subsection*{Book chapters% 论文集的文章
}

\noindent \years{\hspace{0.1in}。}``On the Manchu Names for Grasshoppers, Locusts, and a Few Other Bugs in the Seventeenth and Eighteenth Centuries,'' for inclusion in \emph{Insect Histories of East Asia}, edited by David A. Bello and Daniel Burton-Rose (chapter finished)

\noindent \years{\hspace{0.1in}。}``Manchu, Mandarin, and the Politicization of Spoken Language in Early to Mid-Qīng China,'' for inclusion in \emph{Language Diversity in the Sinophone World} (chapter finished; see above for the book) 

\end{exclude}


\section*{Grants, honors \& awards% 获奖及荣誉
}\phantomsection \label{GrantsHonorsAndAwards}
\noindent \years{2019}Research Project for Junior Researchers Grant, Ministry of Science and Technology, \textsc{r.o.c.} (Taiwan) (1 year, worth \textsc{us}\$22,000)
\begin{quote}Project title: ``Manchu Language Standardization and Qing Institutions During the Eighteenth Century''\end{quote}

\noindent \years{2019}Initial Employment Academic Research Grant (40\% of salary for 2 years)% 新聘學術研究獎金,\\
, Academia Sinica\\
\noindent \years{2018}Temporary Position for Principal Investigator (\emph{Eigene Stelle}), German Research Foundation (\textsc{dfg})\\ (36 months, worth €267,000; declined)
\begin{quote}Project title: ``China's Lost Language: The Vernacular of Northeast Asia under Inner Asian Rule, c. 1000--1644''\end{quote}

\years{2015}Marjory Chadwick Buchanan Dissertation Prize, East Asian Studies Department,\\ Princeton University\\
\years{2014}Seoul National University International Center for Korean Studies Archives Travel Grant\\
\years{2013}1\textsuperscript{st} Prize for Conference Paper, Zhongguo Renmin Daxue Qingshi Yanjiu Suo Di Ba Jie Qingnian Xuezhe Luntan 中国人民大学清史研究所第八届青年学者论坛\\
\years{2013--14}Chinese Government Scholarship\\
\years{2011}Tsang Yee and Wai Kwan Chan So P*71 Fellowship for Chinese History and Culture,\\ Princeton University\\
\years{2010--15}University Fellowship, Princeton University\\
\years{2010}Merit grant from the Center for Human Values, Princeton University\\
\years{2010}Director's Scholarship, Beloit College\\
\years{2010}Stockholm University Scholarship for Student Mobility\\
\years{2010}Global Exchange Scholarship, Hankuk University of Foreign Studies\\
\years{2008--09}\textsc{hsbc} Scholarship, School of Oriental and African Studies\\
\years{2007--08}Bilateral scholarship for study in China, Swedish Institute

%\begin{exclude}

\section*{Presentations% 报告
}

\subsection*{Invited talks% 邀请讲座
}

\noindent \years{2019}``The Creation of a Manchu Vocabulary for Plants and Animals in the Eighteenth Century,'' as part of the graduate seminar ``Histories of Natural History in East Asia: An Overview,'' Princeton University, April 18

\noindent \years{2018}``Leibniz's Dream of a Manchu Encyclopedia and Kangxi's \emph{Mirror}, 1673--1708,'' %Lichtenberg-Kolleg, 
Universität\\ Göttingen, June 13

\noindent \years{2018}```Carts Have Standard Gauges and Documents Standard Script': Plurality and Unity of Writing in the Qing Empire (1644--1911) and Beyond,'' Freie Universität Berlin, February 13 %seminar series "mythographies of scripts and alphabets

\noindent \years{2018}``Following Fuxi or Qubilai Khan? Origins of Written Manchu Before 1644 and Its Reimagining Thereafter,'' Freie Universität Berlin, February 6 %seminar series "mythographies of scripts and alphabets

\noindent \years{2018}``Leibniz, the Kangxi Emperor, and the Manchu Dictionary,'' International Campus, Zhejiang University, January 5

\noindent \years{2017}``Ordning i manchuiska ordböcker, från uppkomsten i sextonhundratalets Peking till den sällsamma vidareutvecklingen i det tidiga artonhundratalets Edo'' (The story of Manchu lexicographic arrangement from its creation in seventeenth-century Beijing to its curious dénouement in Edo in the 1820s), %Lund Circle of East Asian Linguistics, 
Lund University, December 6

\noindent \years{2017}``Kineserna som lärde sig manchuiska: Soldater, ämbetsmän och de första läroböckerna i Qing-dynastins språk'' (The Chinese who learned Manchu: Soldiers, officials, and the first textbooks in the language of the Qing dynasty), %Måndagsföreläsning, 
Stockholm University, October 9

\noindent \years{2017}``The Emperor is Listening: Spoken Language and Officialdom in Manchu China, 1644--1795,'' Universität Heidelberg, June 20

\noindent \years{2017}``Manchu, Mandarin, and the Politicization of Spoken Language in Early- to Mid-Qing China,'' Tel Aviv University, May 23

\noindent \years{2017}``When `Abundance Is a Fault': The Manchu Script and the Search for Pedagogical Simplicity in China and Europe, 1670--1738,'' University of Utah, April 12

\noindent \years{2016}``A Cultural History of The Manchu Script: Lists, Grids, and Metal Type in China and Europe, 1680s--1780s,'' Universität Zürich, September 29

\noindent \years{2016}``Scripts, Statecraft, and the Place of Manchu in the Vision of a Chinese Empire,'' %Ostasienwissenschaftliches Mittagsforum, 
Ruhr-Universität Bochum, July 13

\noindent \years{2015}``L'écriture mandchoue en tant que syllabaire ou alphabet : Une tentative de grammatologie comparée sino-européenne'' (The Manchu script as syllabary or alphabet: An attempt at a comparative sino-european grammatology), as part of the graduate seminar ``Histoire culturelle de la Chine (\textsc{xvi}\textsuperscript{e} siècle--\textsc{xix}\textsuperscript{e} siècle) : livres, éditeurs et publics,''
École des hautes études en sciences sociales, December 17

\noindent \years{2015}``Manchuiska eller mandarin? Qing-imperiets språk i Korea och Japan'' (Manchu or Mandarin? The languages of the Qing empire in Korea and Japan), %Bernhard Karlgren Seminar Series, 
University of Gothenburg, October 15

\subsection*{Conference presentations% 研讨会的报告
}

\noindent \years{2019}``The Qianlong Emperor's Criticism of Manchu Language Use,'' Languages \& Scripts in China: New Directions in Information and Communications History, Columbia University, April 19

\noindent \years{2018}``Joshua Marshman Reads the \emph{Kangxi zidian} Rhyme Tables,'' History of Science Society Annual Meeting, Seattle, November 1--4 %Nov 3 for my paper

\noindent \years{2018}``Language Reform and Manchu Usage in the Eighteenth Century: Some New Sources,'' 22\textsuperscript{nd} Biennial Conference of the European Association for Chinese Studies, Glasgow, August 29--Sept.\ 1 %September 1

\noindent \years{2018}``Was There `Evidential Learning' (\emph{kaozheng xue}) in Manchu? A Look at Language Studies at the Qianlong
Court,'' Multilingual and Multimedia Translation in Qing China, Berkeley, June 1--2 %June 2

\noindent \years{2018}``Allusion and Empire: Banihûn's and Pu-gong's \emph{Manchu-Chinese Literary Ocean} (1821),'' Nineteenth-Century Lexicography Conference, Stanford, April 6--7 %April 7

\noindent \years{2018}``Grammatology and the Manchu Script in Early Nineteenth-Century Paris: Langlès, Rémusat, Klaproth,'' Association for Asian Studies Annual Conference, Washington, D.C., March 22--25%March 22 for my paper

\noindent \years{2018}``Guan yu \emph{Guanhua zhinan} Man yuwen yiben jiqi lishi beijing'' 关于《官话指南》满语文译本及其历史背景 (On the Manchu-language translation of \emph{Guanhua zhinan} [A guide to Mandarin] and its historical background), International Conference on Manchu and Sibe: Language, History, and Culture 满族·锡伯族语言历史文化国际研讨会, Changchun%Northeast Normal University 东北师范大学
, March 17--18 %March 17 for my paper

\noindent \years{2017}``Gottlieb Bayer's (1694--1738) Alphabetic Decipherment of the Manchu Script,'' 2017 Zhongyang Yanjiu Yuan Ming-Qing Yanjiu Guoji Xueshu Yantao Hui 2017中央研究院明清研究國際學術研討會, Taipei, December 18--20 %Dec 19 for my paper

\noindent \years{2017} ``Xiong Shibo's Study of Manchu Phonology at the Turn of the Eighteenth Century,'' The Making of the Humanities \textsc{vi}, Oxford, September 28--30 %Sep 29 for paper, U of Oxford, Somerville college

\noindent \years{2017}``A Controversy Over the Manchu Script,'' 14\textsuperscript{th} International Conference on the History of the Language Sciences, Paris, August 28--September 1 %August 30

\noindent \years{2017}``A Guangxu Renaissance? Manchu Language Studies in the Late Qing and Their Republican Afterlife,'' Rethinking Time in Modern China: A Sinological Intervention, Tel Aviv, May 14--16 %May15

\noindent \years{2016}``A Cultural History of Manchu,'' Sinophone Studies: New Directions, Cambridge, Mass., October 14--15 %Oct 14

\noindent \years{2016}``Leibniz's Hopes for a Manchu Encyclopedia and the Qing Imperial \emph{Mirror} (\emph{han-i araha manju gisun-i buleku bithe}) of 1708,'' 21\textsuperscript{st} Biennial Conference of the European Association for Chinese Studies, Saint Petersburg, August 23--27 %August 25

\noindent \years{2016}``Louis-Mathieu Langlès and the Manchu Moveable Type,'' New Directions in Manchu Studies, Ann Arbor, May 6--7

\noindent \years{2016}```Shooting Characters': A Phonological Game and Its Uses in Late Imperial China,'' Association for Asian Studies Annual Conference, Seattle, March 31--April 3

\noindent \years{2016}``The Rule of a Uniform Script,'' American Comparative Literature Association Annual Meeting, Cambridge, Mass., March 17--20 %18

\noindent \years{2015}``Multilingual Lexicography in Beijing, Seoul and Edo Following the Qing Conquest of Inner Asia,'' 14\textsuperscript{th} International Conference on the History of Science in East Asia, Paris, July 6--10 %7

\noindent \years{2014}``On Alphabetical Order in Manchu Dictionaries of the Qīng Period,'' Association for Asian Studies Annual Conference, Philadelphia, March 27--30 %March 29

\noindent \years{2013}``\emph{Shi'er zitou} yu Qingdai Manwen yuxue'' 《十二字头》与清代满文语学 (The ``Twelve Heads'' and Manchu Language-Studies in the Qing Period), Zhongguo Renmin Daxue Qingshi Yanjiu Suo Di Ba Jie Qingnian Xuezhe Luntan 中国人民大学清史研究所第八届青年学者论坛, Beijing, December 29

\noindent \years{2013}``Organizing Dictionaries in Late Imperial China,'' Dictionary Society of North America 19\textsuperscript{th} Biennial Conference, Athens, Ga., May 22--25

\noindent \years{2013}``Alphabetic Principles and Chinese Characters in Qīng Dictionaries,'' 22\textsuperscript{nd} Annual Graduate Student Conference on East Asia, Columbia University, New York, February 15--16

\noindent \years{2012}``Multilingual Phonology in the Expanding Qīng Empire, 1683--1787,'' China Undisciplined, \textsc{ucla}, Los Angeles, May 18--19

\noindent \years{2011}``Knowledge of Inner Asia in Chosŏn Korea,'' History, Memory and the Politics of Memorialization in Contemporary Korea, Leiden, October 24--27
%\end{exclude}



\subsection*{Campus talks}

\noindent \years{2019}``Zhiwu yu dongwu zai Manwen limian de mingcheng'' 植物與動物在滿文裡面的名稱 (Names of plants and animals in written Manchu), Western Learning and China Research Group colloquium, Institute of Modern History, Academia Sinica, June 14

\noindent \years{2019}``Joshua Marshman (1768-1837) zai Yindu de Zhongguo yuwen xue'' \ldots \ 在印度的中國語文學 (Joshua Marshman's Chinese philology in India), Western Learning and China Research Group colloquium, Institute of Modern History, Academia Sinica, February 14

\noindent \years{2015}``The European Invention of the Manchu Alphabet,'' Department \textsc{iii} colloquium, Max Planck Institute for the History of Science, December 1


%\hrule
\section*{Teaching experience% 教学经验
}
\noindent \years{2017}``Linguistic Spaces in Early Modern Asia,'' graduate seminar in three sessions, Tel Aviv University% 特拉维夫大学
, May 21--24

\noindent \years{2014}Teaching assistant for ``History of East Asia to 1800,'' Departments of History and East Asian Studies, Princeton University% 普林斯顿大学
, fall semester

\section*{Other professional activities% 其他学术活动
}

\noindent \years{2018}Panel organizer, ``Asia and the Global Origins of the Social Sciences, 1700--1900,'' History of Science Society Annual Meeting, Seattle, November 1--4 %nov 3 for our panel

\noindent \years{2015--17}Co-organizer (with Glenn Most and Dagmar Schäfer), ``Thinking in Many Tongues,'' reading seminar meeting four times per year during two years, Max Planck Institute for the History of Science, Berlin

\noindent \years{2017}Discussant, ``Translating Medicine in the Pre-modern World: Knowledge and Practice,'' conference held at the Max Planck Institute for the History of Science, Berlin, June 23--24

\noindent \years{2016}Co-organizer (with Fresco Sam-Sin), ``Digital Manchu Lexicography: A First Meeting,'' one-day event held at the Max Planck Institute for the History of Science, Berlin, December 16

\noindent \years{2016}Panel organizer, ``Sound and Script: Phonological Scholarship and Intellectual Life in Early Modern East Asia,'' Association for Asian Studies Annual Conference, Seattle, March 31--April 3





\section*{Service to the field% 对学术界的服务
}

\noindent %\years{2018}
Co-editor (\emph{Saksaha}, 2019--present); Reviewer of conference abstracts (22\textsuperscript{nd} Biennial Conference of the European Association for Chinese Studies; Colloque \textsc{shesl-htl} 2019); %\years{2016--18}Occasional reviewer for the journal \emph{Saksaha: A Review of Manchu Studies}
Reviewer of article and book-chapter manuscripts (\emph{Late Imperial China}, \emph{Saksaha}, Brill)

\begin{exclude}
%\hrule
\section*{Languages (with formal instruction only)% 语言
}

\begin{tabular}{l l}
\begin{tabular}{l}
Swedish (fluent)\tabularnewline
English (fluent)\tabularnewline
French (fluent)\tabularnewline
Standard Mandarin (fluent)\tabularnewline
Classical Chinese (reading)\tabularnewline
German (reading; intermediate speaking)%\tabularnewline
\end{tabular}
&
\begin{tabular}{l}
Korean (reading)\tabularnewline
Japanese (reading)\tabularnewline
Manchu (reading)\tabularnewline
Classical Mongolian (elementary)\tabularnewline
Latin (elementary)\tabularnewline
Russian (elementary)\tabularnewline \end{tabular}%\tabularnewline
\end{tabular}
%\hrule
\end{exclude}

%\end{exclude}

\begin{exclude}
\section*{References 证明人} %Available upon request

\noindent \begin{tabular}{@{} l l}
\noindent \begin{tabular}{@{} l }
\noindent
Professor \textbf{Dagmar Schäfer}\tabularnewline
Director, Department \textsc{iii}\tabularnewline
Max Planck Institute for the History of Science\tabularnewline
Boltzmannstr.\ 22\tabularnewline
14195 Berlin\tabularnewline
Germany\tabularnewline
Phone: (+49) 030 22667-0\tabularnewline
Email: \href{mailto:dschaefer@mpiwg-berlin.mpg.de}{dschaefer@mpiwg-berlin.mpg.de}\tabularnewline
\end{tabular}
&
\noindent \begin{tabular}{@{} l }
\noindent
Professor \textbf{Benjamin A.\ Elman}\tabularnewline
East Asian Studies and History, Emeritus\tabularnewline
Princeton University\tabularnewline
208 Jones Hall\tabularnewline
Princeton, \textsc{nj}\ 08544\tabularnewline
\textsc{usa}\tabularnewline
Phone: (+1) 609-258-4287\tabularnewline
Email: \href{mailto:elman@princeton.edu}{elman@princeton.edu}\tabularnewline
\end{tabular}
\tabularnewline \tabularnewline
\noindent \begin{tabular}{@{} l }
\noindent
Professor \textbf{Wolfgang Behr}\tabularnewline
Chair for Traditional China\tabularnewline
Universität Zürich\tabularnewline
Zürichbergstr.\ 4\tabularnewline
8032 Zürich\tabularnewline
Switzerland\tabularnewline
Phone: (+41) 44 634 31 80\tabularnewline
Email: \href{mailto:wolfgang.behr@aoi.uzh.ch}{wolfgang.behr@aoi.uzh.ch}\tabularnewline
\end{tabular}
&
\noindent \begin{tabular}{@{} l }
\noindent
Professor \textbf{Jing Tsu}\tabularnewline
\textsc{ealc} \& Comparative Literature\tabularnewline
\textsc{p.o.} Box 208206\tabularnewline
Yale University\tabularnewline
New Haven, \textsc{ct}\ 06520-8206\tabularnewline
\textsc{usa}\tabularnewline
Phone: (+1) 203-432-3426\tabularnewline
Email: \href{mailto:jing.tsu@yale.edu}{jing.tsu@yale.edu}\tabularnewline
\end{tabular}
\tabularnewline \tabularnewline
%\end{tabular}
%
%\begin{exclude}
\noindent \begin{tabular}{@{} l }
\noindent
Professor \textbf{Susan Naquin}\tabularnewline
History and East Asian Studies, Emerita\tabularnewline
Princeton University\tabularnewline
113 41 William St.\tabularnewline
Princeton, \textsc{nj}\ 08544\tabularnewline
\textsc{usa}\tabularnewline
Phone: (+1) 609-258-5801\tabularnewline
Email: \href{mailto:snaquin@princeton.edu}{snaquin@princeton.edu}\tabularnewline
\end{tabular}
&
\noindent \begin{tabular}{@{} l }
\noindent
Professor \textbf{Nicola Di Cosmo}\tabularnewline
East Asian Studies\tabularnewline
Institute for Advanced Study\tabularnewline
1 Einstein Drive\tabularnewline
Princeton, \textsc{nj}\ 08540\tabularnewline
\textsc{usa}\tabularnewline
Phone: (+1) 609-734-8337\tabularnewline
Email: \href{mailto:ndc@ias.edu}{ndc@ias.edu}\tabularnewline
\end{tabular}
\end{tabular}
%\end{exclude}
%
\end{exclude}
%\vspace{1cm}
\vfill{}
%%\hrulefill
\begin{center}
{\scriptsize  Last updated: \today\- •\- 
% ---- PLEASE LEAVE THIS BACKLINK FOR ATTRIBUTION AS PER CC-LICENSE
Typeset in \href{http://nitens.org/taraborelli/cvtex}{
%\fontspec{Times New Roman}
\XeTeX }%\\
% ---- FILL IN THE FULL URL TO YOUR CV HERE
%\href{http://nitens.org/taraborelli/cvtex}{http://nitens.org/taraborelli/cvtex}
}
\end{center}
\end{document}